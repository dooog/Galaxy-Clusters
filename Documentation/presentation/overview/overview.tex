\begin{frame}[allowframebreaks]
\frametitle{Traditional Image Processing \\
vs. High Energy Astronomy}
High energy astronomers \alert{do not} use exactly the same concepts as traditional image processing
%\\~\\ 
%Instead, they have their own \emph{analogues}: 
\begin{center}
  \begin{minipage}{0.75\textwidth}
\begin{block}{\centering Image Processing $\Leftrightarrow$ X-Ray Astronomy}
\begin{center}
      \begin{tabular}{ccccc}
      $\langle x, y\rangle$ & \ \ \ & $\langle R, G, B\rangle$ & \ \ \ & $\alpha$\\
      $\Updownarrow$  & & $\Updownarrow$  &  & $\Updownarrow$ \\
      $\langle \phi, \delta\rangle$ &  & \textup{keV} &  & \textup{photon counts}
      \end{tabular}
\end{center}
\end{block}
\end{minipage}
\end{center}

\framebreak

Where:
\begin{itemize}
\item $\langle \phi,\delta\rangle$ are \emph{spherical coordinates}
	\begin{itemize}
	\item $\phi$ is latitude
	\item $\delta$ is longitude
	\end{itemize}
\item keV are \emph{kilo-electron-Volts}; 1 keV $\approx 1.6\times10^{-16}$ joules
	\begin{itemize}
	\item High energy astronomers look at $0.2$ keV -- $10$ keV
	\item In terms of wavelength ($\lambda$), the range is $6.2$ nm to $124$ pm
	\item By comparison, optical light ranges from $390$ nm (violet) to $750$ nm (red) -- that is from $0.0016$ keV to $0.0031$ keV
	\end{itemize}
\item The ``photon count'' is a direct measure of how many photons were detected by a CCD instrument during an exposure
\end{itemize}
\end{frame}
\subsection{Celestial Coordinate Systems}
\begin{frame}
\frametitle{Celestial Coordinate Systems}
Astronomers commonly use of \alert{three} celestial coordinate systems:
\begin{itemize}
\item Equatorial Coordinates
\item Ecliptic Coordinates
\item Galactic Coordinates
\end{itemize}
\end{frame}
\begin{frame}
\frametitle{Equatorial Coordinates}

\begin{columns}[c]
\column{0.4\textwidth}
Practical \TeX\ 2005\\
Practical \TeX\ 2005\\
Practical \TeX\ 2005
\column{0.4\textwidth}

\end{columns}
\end{frame}
