\subsection{Retrieving Archive Observation Data}
\subsection{Agglomerating Observations by Position}

\begin{frame}[allowframebreaks]
\frametitle{Hierarchical Agglomerative Clustering Overview}
\begin{itemize}
	\item \emph{Hierarchical agglomerative clustering} successively merges pairs of clusters with their \emph{closest} neighbor, based on a measure $D$, until there is nothing left to merge
	\item We use \emph{complete-linkage} as our measure -- that is, we say
\begin{eqnarray*}
  D (X, Y) & = & \underset{y \in Y}{\underset{x \in X}{\max}} (d (x, y))
\end{eqnarray*}
	Where:
	\begin{itemize}
		\item $d(x,y)$ is some distance measure of $x$ and $y$
		\item $X$ and $Y$ are clusters
	\end{itemize}
	\item Efficient $O(n^2)$ algorithm used:\\
	{\small \url{http://math.stanford.edu/~muellner/fastcluster.html}}
\end{itemize}
\framebreak
	The same technique is used in \emph{computational phylogenics}\footnote{Source: \url{http://upload.wikimedia.org/wikipedia/commons/1/11/Tree_of_life_SVG.svg}}
\begin{figure}[h,b]
\centering
\includegraphics[width=0.5\textwidth]{images/Tree-of-life}
\end{figure}
\end{frame}
	
\begin{frame}
\frametitle{Great Circle Distance}
	We define the distance $d(pt_1,pt_2)$ between two points in spherical coordinates by their \emph{Great-circle distance}, which is given by:
\begin{block}{Vincenty's Formula}
\includegraphics[width=\textwidth]{flare-detection/Vincenty}
\end{block}

Where:
\begin{itemize}
\item $\Delta \delta$ is $\delta_s-\delta_f$
\item Implemented using \texttt{atan2} in Python, avoiding numerical instability and divide-by-zero errors
\end{itemize}
\end{frame}
