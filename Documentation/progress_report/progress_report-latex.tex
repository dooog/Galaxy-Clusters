\documentclass{amsart}
\usepackage{geometry,hyperref,calc,capt-of,ifthen}
\geometry{letterpaper}

%%%%%%%%%% Start TeXmacs macros
\newcommand{\tmem}[1]{{\em #1\/}}
\newcommand{\tmtextit}[1]{{\itshape{#1}}}
\newcommand{\tmtexttt}[1]{{\ttfamily{#1}}}
\newcommand{\tmfloatcontents}{}
\newlength{\tmfloatwidth}
\newcommand{\tmfloat}[5]{
  \renewcommand{\tmfloatcontents}{#4}
  \setlength{\tmfloatwidth}{\widthof{\tmfloatcontents}+1in}
  \ifthenelse{\equal{#2}{small}}
    {\ifthenelse{\lengthtest{\tmfloatwidth > \linewidth}}
      {\setlength{\tmfloatwidth}{\linewidth}}{}}
    {\setlength{\tmfloatwidth}{\linewidth}}
  \begin{minipage}[#1]{\tmfloatwidth}
    \begin{center}
      \tmfloatcontents
      \captionof{#3}{#5}
    \end{center}
  \end{minipage}}
%%%%%%%%%% End TeXmacs macros

\begin{document}

\title{Progress Report}
\author{Matthew P. Wampler-Doty\\
\texttt{mpwd@w-d.org}}
\maketitle

\begin{abstract}
  In this report, we detail efforts to produce an automated, large scale
  survey of distant ($z \in [.05, .3]$) galaxy clusters imaged by XMM,
  Chandra, and the Hubble Space Telescope, in search of X-Ray tidal flares. \
  This research is follow up to work first presented in
  {\cite{maksym_constraining_2010,maksym_tidal_2010}}.
\end{abstract}

\section{Overview}

There has been considerable attention to X-Ray tidal flare events, in large
part due to exciting discoveries by the research team behind the SWIFT
satellite {\cite{bloom_possible_2011,metzger_afterglow_2011}}. \ In our
research, we study archival data from Chandra and XMM-Newton, searching for
signs of relativistic outflows from tidal disruption events in distant,
previously inactive galactic nuclei. \ Our research builds on previous work
carried out in {\cite{maksym_tidal_2010}}; most of our efforts have been
focused on developing automated processes for carrying out massive data
analysis. \ The objective of our research is to try to estimate the universal
rate in which these events occur. \ This research is intended as a follow up
to {\cite{maksym_constraining_2010,maksym_tidal_2010}}.

We detail our efforts in the subsequent sections. \ In
{\S}\ref{Agglomeration}, we go over our method for discovering and grouping
observations for future analysis. \ In {\S}\ref{Data_Retrieval}, we detail our
methods of automated data retrieval and mention what we have already done. \
In {\S}\ref{analysis}, we discuss progress made so far in automated data
analysis. \ In {\S}\ref{future} we outline the road ahead, detailing the work
left to accomplish.

The code written primarily utilizes GNU make and Python version 2.6+. \ It is
provided for free on the web, for the inspection of any interested researcher:

\begin{center}
  \href{https://github.com/xcthulhu/Galaxy-Clusters}{https://github.com/xcthulhu/Galaxy-Clusters}
\end{center}

\section{Observation Agglomeration and Classification}\label{Agglomeration}

\subsection{HEASARC Queries\label{heasarc}\label{ACIS}}

The first step in our analysis the automated retrieval of all of the entire
catalog of Chandra and XMM legacy observations using the HEASARC database. \
We make use of python's
\href{http://docs.python.org/library/urllib.html}{\tmtexttt{urllib}} and
\href{http://docs.python.org/library/urllib2.html}{\tmtexttt{urllib2}} to
carry out automated queries to a NASA's CGI script for interfacing this
database{\footnote{NASA's CGI script for interfacing with HEASARC, using POST
methods, can be found here:
\href{http://heasarc.gsfc.nasa.gov/db-perl/W3Browse/w3query.pl}{http://heasarc.gsfc.nasa.gov/db-perl/W3Browse/w3query.pl}}}.
\ Some additional filtering is necessary afterwords - in particular,
observation IDs corresponding to planned observations and unreleased data are
discarded, as well as all observations within $15'$ of the galactic plane, and
only Chandra observations using the ACIS are considers.

Once all of the observations have been acquired, we face the following
difficulty: the labels assigned to objects are not strictly consistent between
missions. \ It is not feasible to use the tags provided by researchers to
classify observations of the same object. \ Our solution is to use
\tmtextit{hierarchical agglomerative clustering} to classify groups of
observations geometrically by position in the sky, eliminating the need rely
on labels.

\subsection{Fast Complete Linkage Hierarchical Agglomerative
Clustering\label{fastagglomeration}}

Hierarchical agglomerative clustering is a technique commonly employed in
\tmtextit{computational phylogenetic tree} research in evolutionary ecology
{\cite{press_section_2007}}. \ We repurpose one of these algorithms for
classifying groups of observations by position.

Abstractly, hierarchical agglomerative clustering algorithms compute a tree
representing hierarchical relationships between points in some set. In
agglomerative clustering, each point is grouped with its nearest neighbor
according to some metric and rule system. \ The system then recurses, only now
over points and previously computed groups of points, building a new stage of
the tree. The process ends when all of the points are grouped together. A
terminal node of the resulting tree represents one the original points, while
branches represent groups of points. We always use the {\tmem{complete
linkage}} rule in our applications, where groups of points in the tree are
labeled with a value representing the maximum distance the points are apart
according to some metric.

As of December 11, 2011 HEASARC reports that there are 12870 suitable archival
observations between XMM and Chandra. We have found that given the number of
observations, agglomerative clustering is intractable without a fast
algorithm. The algorithm we use{\footnote{We use the \tmtexttt{fastcluster}
python module, available here: \href{http://math.stanford.edu/\~{
}muellner/fastcluster.html}{http://math.stanford.edu/\~{
}muellner/fastcluster.html}. The code implemented runs in $\mathcal{O} (n^2)$
time (where $n$ is the number of points being clustered). \ Other library
implementations of agglomerative clustering use algorithms that often run in
$\mathcal{O} (n^3$) time; this includes the \tmtexttt{scipy} python module and
\tmtexttt{matlab}.}} is given in {\cite{defays_efficient_1977}}.

After computing the hierarchical agglomeration, we find each group of
observations separated by at most $8'$ and output the observation IDs of that
group to a designated file. \ As of December 11, 2011 the system computes 5846
groups of observations $8'$ apart.

Figure \ref{fig1} depicts the all of the groups of observations found by the
algorithm that have 5 or more entries. The size of each node, designating a
group of observations, is proportional to the $\log (n)$, where $n$ is the
number of observations in that group.

\begin{center}
  \tmfloat{h}{small}{figure}{FIXME}{Groups of observations in the XMM and
  Chandra archives}\label{fig1}
\end{center}

\subsection{NED Queries}\label{ned}

Not every group is suitable for analysis, as we are interested in galaxy
clusters in particular. \ To find only those groups of observations
corresponding to a galaxy cluster, we use Cal-tech's NED database as the basis
of further processing. For each agglomeration of observations, we query NED to
find all of the galaxy clusters within the vicinity. \ An arbitrary
observation $O$ from each group of observations is selected, and a query is
sent to NED for all galaxy cluster objects in a $15'$ radius of $O$. \ Queries
are automated using a python script using \tmtexttt{urllib} and
\tmtexttt{urllib2}{\footnote{Cal-tech's NED database can be accessed using GET
methods and the following CGI script:
\href{http://ned.ipac.caltech.edu/cgi-bin/nph-objsearch}{http://ned.ipac.caltech.edu/cgi-bin/nph-objsearch}}},
just as in the case of HEASARC as discussed in {\S}\ref{heasarc}.

After all of the NED queries have been computed, the results are filtered such
that only galaxy clusters with $z$-values in $[.05, .3]$ are kept. \ Any
agglomeration which does not have a suitable galaxy cluster in its vicinity is
discarded.

Figure \ref{fig2} depicts the results of this further filtering carried out on
the groups of observations found in {\S}\ref{fastagglomeration}. \ In
addition, this figure uses transparency to reflect $z$ values, with the most
transparent reflecting a $z = .05$ while the least transparent reflects $z =
.3$.

\begin{center}
  \tmfloat{h}{small}{figure}{FIXME}{Groups of observations of galaxy
  clusters}\label{fig2}
\end{center}

\section{Data Retrieval}\label{Data_Retrieval}

In this section we detail how data is retrieved corresponding to the
previously grouped observations.

After each group of suitable observations is found, we retrieve data files for
each observation in those groups. \ In addition, we also retrieve HST data in
the vicinity of each group of observations, in the interest of finding optical
follow-ups to our found sources.

To date, approximately 615 GB of data have been downloaded as part of our
survey.

\subsection{Chandra}

Each Chandra observation is retrieved using the
\tmtexttt{download\_chandra\_obsid} script, using the CIAO{\footnote{CIAO is
available from the following website:
\href{http://cxc.harvard.edu/ciao/}{http://cxc.harvard.edu/ciao/}}} data
analysis software for chandra.

\subsection{XMM}

A python script was written to automate the acquisition of XMM-Newton data. \
Specifically, we automated interaction with the XSA online interface to XMM
archive data{\footnote{The XSA website's CGI script can be found here:
\href{http://xsa.esac.esa.int:8080/aio/jsp/product.jsp}{http://xsa.esac.esa.int:8080/aio/jsp/product.jsp}}}.
\ After querying, the XSA CGI-script returns an HTML page containing an FTP
address of the prepared data. \ Regular expressions are used to parse this
webpage to acquire this FTP address. \ The data is then retrieved using the
UNIX \tmtexttt{wget} facility.

\subsection{Hubble}

For each group of observations, all of the Hubble observations in the vicinity
are found by querying the Space Telescope Science Institute{\footnote{A
catalog of archival HST observations can be accessed using GET methods and the
following CGI script:
\href{http://archive.stsci.edu/hst/search.php}{http://archive.stsci.edu/hst/search.php}}}.
\ After a catalog of HST observations is found for a given group of
observations, those HST observations making use of the WFPC, WFPC2, WFC3, and
ACS instruments are selected. \ This is done because these instruments provide
the best optical imaging. \ From each entry, a URL is retrieved containing the
archival data{\footnote{Data products are found using the specifications found
in
\href{http://www.stsci.edu/instruments/wfpc2/Wfpc2\_dhb/wfpc2\_ch22.html}{http://www.stsci.edu/instruments/wfpc2/Wfpc2\_dhb/wfpc2\_ch22.html}
. \ Files are retrieved using GET methods and the following CGI script:
\href{http://www.stsci.edu/instruments/wfpc2/Wfpc2\_dhb/wfpc2\_ch22.html}{http://www.stsci.edu/instruments/wfpc2/Wfpc2\_dhb/wfpc2\_ch22.html}}}
using the UNIX command \tmtexttt{wget}.

\section{Data Analysis}\label{analysis}

In this section we detail how the data we have retrieved is analyzed. \ To
date, we have focused on chandra data analysis. \ Note that as we mentioned in
{\S}\ref{ACIS}, we only look at the Chandra ACIS instrument. \ This due to the
fact that this is the only instrument on Chandra with sub-arcsecond precision.

\subsection{Chandra Reprocessing}

Due to a number of errors in the preprocessing of the Chandra archival data,
it is necessary to run a reprocessing script before any further data
manipulation is performed{\footnote{A detailed discussion of the reasons for
reprocessing Chandra data files can be found here:
\href{http://cxc.harvard.edu/ciao4.4/threads/createL2/index.html}{http://cxc.harvard.edu/ciao4.4/threads/createL2/index.html}}}.
\ Reprocessing is carried out using the \tmtexttt{chandra\_repro} script from
the CIAO software suite.

\subsection{Chandra Band Extraction}

After each chandra observation has been reprocessed, we extract particular
energy bands from the newly produced \tmtexttt{evt2.fits} event files of
(filtered) CCD events. \ This is done using the CIAO tool
\tmtexttt{dmcopy}{\footnote{The syntax used is \tmtexttt{dmcopy
evt2.fits[energy>$\alpha$,energy<$\beta$] result.fits}, which extracts events
with energy $\eta$ (in electron-Volts)where $\eta \in [\alpha, \beta]$}}. \ We
perform extraction for each of the following bands found in Table \ref{tab1},
defined in {\cite{kim_chandra_2007}}.

\begin{table}[h]
  \begin{tabular}{ll}
    Band & Energy\\
    Broad (B) & 0.3 -- 8 keV\\
    Soft (S) & 0.3 -- 2.5 keV\\
    Harder (H) & 2.5 -- 8 keV\\
    Soft1 (S$_1$) & 0.3 -- 0.9 keV\\
    Soft2 (S$_2$) & 0.9 -- 2.5 keV
  \end{tabular}
  \caption{\label{tab1}Chandra Energy Bands}
\end{table}

\subsection{Chandra Source Detection}

After each band is extracted, we use the CIAO facility \tmtexttt{wavdetect} to
detect point sources. \ This method uses fast wavelet transforms to correlate
Mexican hat wavelets{\footnote{A Mexican hat distribution is the second
derivative of a Gaussian; in the 1 dimensional case it follows
$\frac{2}{\pi^{1 / 4}  \sqrt{3 \sigma}} \left( 1 - \frac{t^2}{\sigma^2}
\right) e^{- \frac{t^2}{2 \sigma^2}} = \frac{\partial^2}{\partial t^2} g$ . \
The higher dimensional Mexican hat wavelet generalizes the 1 dimensional
second derivative to a Laplacian.}} at different scales, selected by the user,
and associates peaks with sources. The waves we correlate have as parameters
$\sigma \in \left\{ 1, \sqrt{2}, 2, 2 \sqrt{2}, 4, 4 \sqrt{2}, 8 \right\}$.

\subsection{Source Agglomeration}\label{source_agg}

After sources are detected for each chandra image, it is necessary to find
common sources across multiple bands and images. \ We employ fast complete
linkage agglomerative clustering, as we introduced in
{\S}\ref{fastagglomeration}, to achieve this. \ After clustering groups no
bigger than $3''$ apart are separated and cataloged.

\section{Future Work}\label{future}

\subsection{Better Chandra Source Detection}

Source detection does not make use of recent innovations in the CIAO software
suite. \ Recent versions of wavdetect use Chandra's point spread function
(PSF) that was originally found by engineers during ground calibration. \
Recently, CIAO can now compute the PSF for each individual image using the
\tmtexttt{mkpsfmap} program. \ The observation-specific PSF can be used by
\tmtexttt{wavdetect} to import source detection.

\subsection{XMM Astrometry Correction}\label{astrometry}

We expect that XMM and Chandra may have some discrepancies between their
astrometry; it is necessary to introduce adjustments to these images to bring
them into alignment. \ We hope to use the \tmtexttt{geotran} matching package
from IRAF{\footnote{The \tmtexttt{geotran} UNIX manual page is available
online here:
\href{http://iraf.noao.edu/scripts/irafhelp?geotran}{http://iraf.noao.edu/scripts/irafhelp?geotran}}}.

\subsection{XMM Band Extraction and Source Detection}

Chandra energy levels $\eta$ may be converted to XMM-Newton energy levels
$\xi$ using the following equation{\footnote{This equation may be determined
by inspecting the equivalent energy levels of the two satellites listed on the
following website:
\href{http://heasarc.gsfc.nasa.gov/W3Browse/all/ic10xmmcxo.html}{http://heasarc.gsfc.nasa.gov/W3Browse/all/ic10xmmcxo.html}}}:
\[ \xi \approx 0.09 \eta^2 + 1.75 \eta - .2 \]
This means that the energy levels for the different bands are slightly altered
from those for Chandra. \ Table \ref{tab2} summarizes these differences,
rounded to 2 significant figures.

\begin{table}[h]
  \begin{tabular}{lll}
    Band & Chandra Energy & XMM Energy\\
    Broad (B) & 0.3 -- 8 keV & $0.3$ -- $8$ keV\\
    Soft (S) & 0.3 -- 2.5 keV & $0.3$ -- 3.6 keV\\
    Harder (H) & 2.5 -- 8 keV & 3.6 -- 8 keV\\
    Soft1 (S$_1$) & 0.3 -- 0.9 keV & 0.3 -- 1.3 keV\\
    Soft2 (S$_2$) & 0.9 -- 2.5 keV & 1.3 -- 3.6 keV
  \end{tabular}
  \caption{\label{tab2}XMM vs. Chandra Energy Bands}
\end{table}

These energy levels are used as parameters for the SAS
\tmtexttt{edetect\_chain} script{\footnote{The SAS \tmtexttt{edetect\_chain}
script documentation can be found here:
\href{http://xmm.esa.int/sas/8.0.0/doc/edetect\_chain/index.html}{http://xmm.esa.int/sas/8.0.0/doc/edetect\_chain/index.html}}},
which plays a similar r\^ole for XMM as \tmtexttt{wavdetect} plays for
Chandra.

\subsection{Source Agglomeration}

After computing sources for XMM, it will be necessary to produce an
agglomeration encompassing both Chandra and XMM detected sources. \ It is
critical that the astrometry correction described in {\S}\ref{astrometry} be
completed prior to this step, otherwise we run we run the risk of error being
introduced by systematic miscalibration. \ We intend to use the same approach
described in {\S}\ref{source_agg}.

\subsection{XSPEC Flux Analysis}

After groups of sources are found for each group of observations, we compute
the flux around each source for each image, using
\tmtexttt{xspec}{\footnote{The documentation for \tmtexttt{xspec} can be found
here:
\href{http://heasarc.nasa.gov/xanadu/xspec/XspecManual.pdf}{http://heasarc.nasa.gov/xanadu/xspec/XspecManual.pdf}}}.
\ Before computing fluxes, it is necessary to find an background region for
each source, and filter out spurious background noise. \ To find background
regions we will use the python package \tmtexttt{kapteyn} to find encompassing
$1''$ circles around each source. After each circle of background events is
found, we break the background region up into quadrants, compute the average
energy and standard deviation for each quadrant, and then throw away quadrants
that deviate more than $3 \sigma$ from the others. \

Once the background regions and calculated, \tmtexttt{xspec} will be used to
compute the source flux by black-body spectrum fitting. \ After fluxes are
found for each group of sources, we will attempt to fit an exponential decay
through time to the group of fluxes. \ Following the work presented in
{\cite{maksym_tidal_2010}}, a point source flux decaying exponentially through
time is construed as evidence of a tidal flare.

\subsection{Estimate Observed Galaxy Count}

As described in the introduction, this research is intended to present a
follow up to {\cite{maksym_constraining_2010}}. \ We intend to compute an
estimate or lower bound on the universal frequency of tidal flares. \ We hope
to use known estimates of galaxy counts around these clusters when possible,
although it may be necessary to automate estimating the number of galaxies.

\subsection{Hubble Optical Analysis}

Once candidate flares are found, we would ideally like to identify an optical
counter-part. \ Given optical counter-parts we could then have candidates
objects for red-shift estimation, which would corroborate our the theory that
the tidal flares are originating from sources within the galaxy cluster. \ As
with the case of Chandra and XMM, it may be necessary to perform astrometry
corrections.

\section{Conclusion}

While a number of hurdles to our research have been overcome, many remain. \
Unfortunately, many of the software utilities we use have not been optimized
for speed, rendering data analysis at the scale we wish to operate at
difficult. \ Fortunately, our infrastructure uses open source and allows for
ad hoc parallelism so any number of computers can be employed without needing
to purchase proprietary software. \ We hope to have completed our survey of
our 75 candidates soon.

\bibliography{bibliography}{}
\bibliographystyle{plain}

\end{document}
